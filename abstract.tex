Background: Methylmercury is one of the most toxic forms of mercury, which severely afflicts the developing fetus. Methylmercury exposure occurs not only through fish consumption but also through rice consumption. However, rice does not contain the same beneficial micronutrients that fish/shellfish has. Therefore, prenatal methylmercury exposure through maternal rice consumption may potentially increase adverse health effects on the developing fetus. 

Objectives: The purpose of this dissertation was to investigate the magnitude of prenatal methylmercury exposure through rice consumption and the relationship between maternal mercury biomarkers and birth outcomes.

Methods: A total of 398 pregnant women were recruited at parturition in a rural area of southern China, where rice was a staple food and mercury contamination was considered minimal. Total mercury and/or methylmercury concentrations were measured in maternal hair and blood, in rice samples from each participant's home, and in freshwater fish tissue purchased from local markets. Additional fish/shellfish mercury levels were obtained from a literature review. Neonatal weight, length and head circumference were recorded shortly after birth and then converted to: birth weight z score, birth length z score, head circumference z score, and ponderal index (\(\text{birth weight} / \text{birth length}^{3}\)). 

Results: The geometric means of hair total mercury, hair methylmercury, and blood total mercury were 0.41 \({\mu}\)g/g, 0.27 \({\mu}\)g/g, and 1.3 \({\mu}\)g/L, respectively, which were comparable to other pregnant cohorts with low-level mercury exposure mainly through fish/shellfish consumption. All three mercury biomarkers were significantly correlated with rice methylmercury intake, but not fish/shellfish methylmercury intake. After adjusting for covariates, both hair total mercury and hair methylmercury were negatively related with birth weight z score. An inverse relationship was also observed between blood total mercury and ponderal index. In contrast, the relationships between mercury biomarkers and other outcome measures were inverse but non-significant.

Conclusions: Our findings confirm the presence of low-level prenatal methylmercury exposure occurring in this cohort, and indicated that rice intake contributed to the prenatal methylmercury exposure, more so than fish/shellfish intake. Also, inverse linear relationships between prenatal methylmercury exposure and birth weight and ponderal index were observed. 
