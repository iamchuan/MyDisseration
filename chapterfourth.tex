\chapter{Summary and Conclusions}

\section{Summary}

Hg is a toxic heavy metal and a global pollutant (National Research Council, 2000). MeHg is one of the most toxic forms of Hg, which biomagnifies along the food chain and accumulates in the human body (National Research Council, 2000). MeHg can cross the placenta and pass through the blood-brain barrier; the developing fetus is the most vulnerable population to the adverse effects of MeHg exposure (Clarkson \& Magos, 2006; Mergler et al., 2007). A few recent studies have shown that maternal exposure to low-level MeHg during pregnancy influenced fetal growth and development (reviewed in (Karagas et al., 2012)). For many people in the world, fish/shellfish is the primary source of MeHg exposure (National Research Council, 2000; Clarkson \& Magos, 2006), however, human dietary MeHg exposure also occurs through rice ingestion (Rothenberg et al., 2014). Rice does not contain the same beneficial nutrients for fetal development as fish/shellfish does, such as long chain polyunsaturated fatty acids, iodine, and selenium. Thus, maternal MeHg exposure mainly through rice ingestion may increase adverse influences on the developing fetus without the nutritional benefits from the consumption of fish/shellfish (Rothenberg et al., 2011). To our knowledge, few studies reporting rice Hg concentrations or rice ingestion and Hg biomarker levels were conducted in pregnant cohorts.

In Chapter 2, we measured the THg and/or MeHg concentrations in maternal scalp hair and whole blood in a cohort of newborns recruited at parturition from an inland rural area of Guangxi, China, where rice is a staple food and Hg contamination was considered minimal. The findings of this chapter confirmed the presence of low-level prenatal MeHg exposure occurring in this birth cohort, and indicated that maternal rice
ingestion during pregnancy contributed to the prenatal MeHg exposure, more so than the consumption of fish/shellfish. Meanwhile, the magnitude of prenatal MeHg in the Daxin cohort was comparable to other pregnant cohorts worldwide with low-level Hg exposure mainly through fish/shellfish consumption.

Furthermore, we examined the associations between socioeconomic characteristics and dietary MeHg intake. The study population of mothers represented a typical rural population of inland regions in Southern China. Similar to previous studies, we found that fish/shellfish consumption was higher for mothers who possessed a higher education level or had a higher household income (Knobeloch et al., 2005; Mahaffey et
al., 2009; Wang et al., 2009; Zhou et al., 2015; Hightower \& Brown, 2011). Notably, we found that mothers who were farmers consumed relatively low levels of fish/shellfish, which may be explained by the limited access to fish/shellfish. This is of interest because the food environment of famers is similar to the ``food desert''. Farmers in this cohort lived far away from the town of Daxin where diverse food (including fish/shellfish) was sold by local markets or groceries. Such a ``food desert'' may potentially affect farmers'
food access and nutritious status, however, fewer studies in China focused on this issue.

In Chapter 3, we examined the associations between maternal Hg biomarkers (i.e. hair THg, hair MeHg, and blood THg) and neonatal outcomes (i.e. birth weight, birth length, head circumference, and ponderal index) in both unadjusted and adjusted linear regression models. Firstly, we observed that birth weight z score was inversely associated with all Hg biomarkers in adjusted linear regression models, but the trends for hair THg
and hair MeHg were significant, while the trend for blood THg was borderline. Our findings were similar to some previous studies, which observed significantly inverse relationships between birth weight and some Hg biomarkers, such as cord blood THg (Ram�n et al., 2009; Lee et al., 2010) or maternal blood THg (Lee et al., 2010; Ou et al., 2015). Secondly, we observed inverse but non-significant adjusted linear relationships
between all Hg biomarkers and birth length z score or head circumference z score, which were consistent with other studies. For example, a few studies have reported that birth length was not related with prenatal Hg exposure in cord blood (Ding et al., 2013; Guo et al., 2013; Lederman et al., 2008; Wells et al., 2016), maternal hair (Drouillet-Pinard et al., 2010; Guo et al., 2013), or maternal blood (Lederman et al., 2008; Ding et al., 2013), in adjusted models. Also, six recent studies which evaluated head circumference (Ding et al.,
2013; Drouillet-Pinard et al., 2010; Gundacker et al., 2010; H. Guo et al., 2009; Lederman et al., 2008; Wells et al., 2016) observed non-significant associations between prenatal low-level MeHg exposure and head circumference in adjusted models. Lastly, the ponderal index was found to be significantly decreased with increasing blood THg in adjusted models, but not with other Hg biomarkers. Our findings were consistent with a Baltimore study, which observed an inverse association of cord blood MeHg and ponderal index after adjusting for potential covariates (Wells et al., 2016).

Moreover, this chapter provided an opportunity to investigate some protective factors related to fetal growth and development, such as fish/shellfish consumption and Se. However, the strength of the relationships between Hg biomarkers and birth outcomes in the Daxin cohort were not affected by the inclusion of fish/shellfish consumption frequency or maternal serum Se.

\section{Strengths and limitations}

A major strength of this study was the recruitment conducted in an area without Hg contamination, which is helpful to understand the general magnitude of MeHg exposure through rice consumption in a rice-eating population and the potential effects on fetal growth. Meanwhile, the sample size of this study is much larger than the previous studies (Maramba et al., 2006; Rothenberg et al., 2013), which improve the statistical power to evaluate the association between prenatal MeHg exposure and birth outcomes. Other strengths included the application of three maternal Hg biomarkers and four birth outcomes in this study, which made our findings more comprehensive and comparative.

There are several limitations in this study. The FFQ for the third trimester was conducted at parturition and was self-reported by the mothers, which might affect the validity of the maternal consumption information due to recall bias or wrong reporting. Moreover, we only measured 13 freshwater fish samples from Daxin local markets and did a literature review to obtain the Hg levels in other types of fish/shellfish. Future studies should measure Hg levels in all types of fish/shellfish included in the FFQ.

\section{Future studies}

Future studies of prenatal MeHg exposure through maternal consumption of rice and birth outcomes should be conducted to confirm these findings. For the present study, we collected the information of exposure and birth outcomes at parturition. Future studies about MeHg exposure during pregnancy could improve the study design. A longitudinal study conducted across the duration of pregnancy would be useful for determining the influence on fetal growth. For example, serial ultrasounds could be used to examine fetal
growth trajectory, and Hg levels in the maternal blood and hair could be measured at different stages during the pregnancy.

Additionally, the issue of ``food desert'' should be considered in future studies. The association between prenatal MeHg exposure and fetal growth may be affected by the maternal nutritional status due to the limited access to diverse or nutritious foods. To date, many studies have reported that the nutrition intake and diet quality before or during pregnancy is associated with fetal growth and development (Institute of Medicine (IOM), 1990; G. Wu, Bazer, Cudd, Meininger, \& Spencer, 2004). For instance, poor nutritious status before or during pregnancy was found to be associated with some adverse birth outcomes, such as small for gestational age (Mitchell et al., 2004) and lower birth weight (Sram, Binkova, Lnenickova, Solansky, \& Dejmek, 2005). Some recent studies reported that neighborhood food environments are associated with human dietary patterns (Walker, Keane, \& Burke, 2010). For example, a US study of adults observed that participants living far away from supermarkets (> 1 mile) were 25 - 46\% less likely to have a healthy dietary pattern than other respondents who did their food shopping within 1 mile of their home after adjustment for confounders (e.g. age, sex, race/ethnicity, and socioeconomic factors) (Moore \& Diez Roux, 2006). It is worth noting that the dietary quality and nutrition intake during pregnancy has a relationship with the food environment where the pregnant women live (Laraia, Siega-Riz, Kaufman, \& Jones, 2004). For example, in a North Carolina study, Laraia et al. observed that the distance to the closest supermarkets was positively associated with the diet quality index for pregnancy after adjusting for confounders (individual characteristics and other food retail outlets) (Laraia et al., 2004). However, to date, there are few studies on the food environment and fetal growth and the results are inconsistent. In a New York study, the authors observed that women living near a supermarket had significantly fewer low birth weight births after controlling for covariates, such as income (Lane et al., 2008). While in a Louisiana study, the authors reported that the local supermarket density was associated with neither the gestational age nor the birth weight-for-gestational age (Farley et al., 2006). Similarity, in a South Carolina study, authors found that either the accessibility or the availability of supermarkets and grocery stores was not associated with birth
outcomes and gestational age (Ma, Liu, Hardin, Zhao, \& Liese, 2015). Therefore, the food access in this study cohort may be a potentially important factor influencing the fetal growth, and should be evaluated in future studies about the association between prenatal MeHg exposure and birth outcome. For example, we can collect the information of individual access to nutritious or diverse foods (e.g. fish/shellfish), such as the distance from a participant's home to the nearest food market and the number of food markets within a 1-mile distance from participant's home. Then, we can evaluate the relationship of food access with MeHg exposure and birth outcomes.

\section{Conclusions}

In summary, these findings of this study confirm that prenatal exposure to MeHg was occurring among newborns born in Daxin County due to maternal consumption of rice during pregnancy. Despite the relatively low-level MeHg concentrations in maternal biomarkers (i.e. hair THg, hair MeHg, and blood THg), we observed significantly inverse linear relationships between birth weight and hair THg and hair MeHg, and also between ponderal index and blood THg in adjusted linear regression models. Future studies are needed to confirm these findings in other pregnant women cohorts who consume rice as a staple food. Given the human exposure to MeHg mainly through rice consumption in rice-eating populations, potential influences on fetal growth and development could have significant implications.














