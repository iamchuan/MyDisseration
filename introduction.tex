\chapter{Introduction}

\section{Background}

Mercury (Hg) is a toxic heavy metal that has been used commercially and medically for centuries (e.g., Hg mining, batteries, fluorescent lamps, dental amalgams, thermometers, and blood pressure cuffs) \citep{who1990mehg,usepa1997hgcongress,clarkson2006toxicology}. During the past several decades, Hg exposure has been becoming an important issue of global health due to its widespread distribution and persistence in the environment, as well as its bioaccumulation through the food chain, with potential adverse influences on humans and ecosystems \citep{who1990mehg,national2000toxicological,clarkson2003toxicology}. In 2013, an international treaty named the Minamata Convention on Mercury was opened for signature and ratification. The aim of this treaty is to protect human health and the environment from anthropogenic production and release of Hg, such activities as mining, fossil fuel combustion, and waste management of products containing Hg \citep{unep2016mcm}. The Minamata Convention on Mercury will go into effect 90 days after it has been ratified by at least 50 nations. To date, 128 countries have signed the Minamata Convention; 29 of these countries have ratified the treaty, including the U.S., which was the first nation to do so \citep{unep2016mcm}.

There are three primary forms of Hg in the environment: elemental Hg (\(\text{Hg}^{0}\)), inorganic mercurial salts (\(\text{Hg}^{2+}\)), and organic Hg [primarily methylmercury (MeHg)]  \citep{usepa1997hgcongress,clarkson2002three}. In nature, \(\text{Hg}^{0}\) is released from the Earth's crust (e.g. soil and rocks) through weathering, from lakes and oceans, and from volcanoes and forest fires \citep{usepa1997hgcongress,clarkson2002three}. However, the main source of \(\text{Hg}^{0}\) is anthropogenic, including coal-burning power plants and incinerators \citep{mason1994biogeochemical}. Once Hg is released into the environment, it becomes highly mobile due to its volatility and cycles in the atmosphere, lithosphere, hydrosphere, and biosphere (U.S. Environmental Protection Agency, 1997; Clarkson, 2002). After atmospheric emission, \(\text{Hg}^0\) is oxidized to \(\text{Hg}^{2+}\), which may return to the earth in rainfall or dry deposition (U.S. Environmental Protection Agency, 1997; Clarkson, 2002). In the aquatic environment, \(\text{Hg}^{2+}\) can be methylated to MeHg by anaerobic microorganisms \citep{parks2013genetic}, which is biomagnified in fish, shellfish, and sea mammals (e.g., whales, seals) \citep{morel1998chemical}. MeHg has a long half-life in aquatic organisms, resulting its bioaccumulation and biomagnification in the aquatic food chain (National Research Council, 2000). 

Fish/shellfish is the predominant source of human dietary MeHg exposure \citep{who1990mehg,usepa1997hgcongress,clarkson2006toxicology}. Recently, MeHg exposure via rice consumption has also been reported by a few studies \citep{feng2007human,rothenberg2011low,li2012rice,davis2014dietary}. Thus, rice consumption may play a significant role in dietary MeHg intake in populations who consume rice as a staple food \citep{rothenberg2011low,rothenberg2013prenatal,rothenberg2014rice}.

\section{Methylmercury: toxicity, source, and exposure}

\subsection{MeHg characteristics}

MeHg is one of the most toxic forms of Hg (National Research Council, 2000). After dietary MeHg exposure, MeHg is absorbed and distributed throughout the body (World Health Organization, 1990). Intracellular mercuric ions attach themselves to the thiol residues of proteins, which blocks the normal bindings of enzymes, cofactors, and hormones \citep{bridges2005molecular}. The central nervous system is the major system impacted by the MeHg exposure (e.g., coordination, the senses of touch, taste, and sight). MeHg also affects the immune system, and genetic and enzyme systems (World Health Organization, 1990; U.S. Environmental Protection Agency, 1997; Clarkson \& Magos, 2006). 

MeHg exposure poses a significant health risk to the developing fetus \citep{clarkson2006toxicology, mergler2007methylmercury,grandjean2010adverse,bose2010mercury}. MeHg readily crosses the placenta and blood-brain barrier \citep{cernichiari1994biological,ramirez2000tagum}. 

\subsection{Biomarkers of MeHg exposure}

Both scalp hair Hg and whole blood Hg have been used as valid biomarkers in studies of human MeHg exposure \citep{clarkson2006toxicology,mergler2007methylmercury}. Total mercury (THg) levels in the whole blood can give an estimate of both MeHg and inorganic Hg exposure \citep{kershaw1980relationship,berglund2005inter,clarkson2006toxicology}. In a study, \cite{kershaw1980relationship} firstly gave participants a single meal of fish (dose of Hg consumed: 18 to 22 \({\mu}\)g/kg body weight. Then they found that the blood THg concentrations reached the peak ranging from 4.7 to 14 hours, while the blood inorganic Hg concentrations did not change significantly. The author indicated that the increase of THg levels in whole blood following the consumption of fish was due to the MeHg intake \citep{kershaw1980relationship}. Blood THg has usually been used as an indicator of MeHg exposure in fish-eating cohorts with an assumption that the inorganic Hg exposure was minimal \citep{schober2003blood,weil2005blood}.

Hair THg can reflect MeHg exposure at all exposure levels, but not inorganic Hg exposure \citep{lindberg2004exposure,berglund2005inter,clarkson2006toxicology}. After exposure, MeHg is first accumulated in the hair shaft; then, a small fraction of MeHg is demethylated to inorganic Hg \citep{berglund2005inter}. Hair has the advantage of being able to reflect the MeHg exposure over the growth period of the hair segment. The length of the hair from the scalp determines the period of MeHg
exposure; the proximal end is associated with recent exposure. Hair Hg has been used to reflect Hg levels throughout the pregnancy in some studies on prenatal MeHg exposure \citep{clarkson2006toxicology}.

\subsection{Profile of prenatal MeHg exposure}

\subsubsection{MeHg poisoning cases}

The outbreak of the Minamata disease in Japan in 1956 was one of the first high-level MeHg poisonings that were documented \citep{kurland1962minamata,harada1995minamata}. An acetaldehyde factory dumped Hg and MeHg into the Minamata Bay, and residents in the area were poisoned by the consumption of fish and shellfish containing high concentrations of MeHg \citep{harada1995minamata}. Another large-scale MeHg poisoning catastrophe happened in rural Iraq in the early 1970s. Hundreds of MeHg poisonings were confirmed to be caused by accidental consumption of homemade bread mistakenly made from seed grain of wheat treated with MeHg fungicide \citep{bakir1973methylmercury}. 

In both tragedies, the offspring were severely affected due to the high levels of prenatal MeHg exposure, and the prenatal period was found to be the period in the life cycle most sensitive to MeHg (reviewed in \cite{clarkson2006toxicology}). Children from the Japan MeHg poisoning were born with severe cerebral palsy-like symptoms (including mental retardation, dysarthria, cerebellar ataxia, deformity of the limbs, primitive reflexes, hypersalivation, disturbance in physical development and nutrition, and even death), while their mothers showed mild or no symptoms of MeHg poisoning \citep{harada1995minamata}. For the Iraq outbreak, a clinical study was conducted on the brains from two full-term infants who were exposed to MeHg in utero and had died shortly after birth \citep{choi1978abnormal}. The researchers found that the cytoarchitecture of the autopsied brain tissue was extensively disrupted \citep{choi1978abnormal}. Moreover, a study reported that the prevalence of delayed walking (defined as not walking until after 18 months of age) in the offspring from the Iraq outbreak increased as the maternal hair Hg levels during pregnancy rose \citep{cox1989dose}. This study also reported that the threshold level of maternal hair Hg for delayed walking was as low as 10 ppm, while the threshold level for paresthesia in adults was 100 ppm, indicating the greater sensitivity of the fetus \citep{cox1989dose}. 

\subsubsection{Studies on Prenatal MeHg exposure through maternal seafood consumption}

Apart from MeHg poisoning cases, for the general population, the main route of chronic, low-dose prenatal MeHg exposure is consumption of fish \citep{national2000toxicological,smith2005fish}. The dominant form of Hg in fish/shellfish is MeHg, accounting for greater than 90\% of the THg (National Research Council, 2000). MeHg is bound to the proteins of fish/shellfish muscle tissues, which are not removed by any cooking procedures \citep{mergler2007methylmercury}. 

There are three significant long-term cohort studies on prenatal MeHg exposure through seafood consumption and infant or child development (New Zealand, Faroe Islands, and Seychelles Islands): 

\underline{New Zealand Study} A birth cohort study examining the relationship between prenatal MeHg exposure through maternal consumption of fish and child neurodevelopment was conducted in New Zealand \citep{kjellstrom1986physical,kjellstrom1989physical,crump1998influence}. During 1977 and 1978, maternal scalp hair samples and a diet questionnaire were collected from 10,970 pregnant women in hospitals on the New Zealand \citep{crump1998influence}. The participants in the New Zealand study consumed marine fish frequently and the fish species included shark containing high levels of MeHg (the maximum: 4 ppm) \citep{clarkson2006toxicology}. Researchers evaluated the relationships between maternal hair Hg concentrations during pregnancy and neurodevelopmental outcomes at age 4 years \citep{kjellstrom1986physical} and at age 6 years \citep{kjellstrom1989physical}, respectively. These studies found that the neurodevelopment in children whose mothers were with high hair Hg concentrations (> 6 ppm) was delayed: at age 4, visual perception, language, memory, and motor function were evaluated using the Denver Developmental Screening Test \citep{kjellstrom1986physical}; at age 6, a group of 26 psychological and scholastic tests were carried out, covering academic attainment, language development, fine and gross motor coordination, and intelligence \citep{kjellstrom1989physical,crump1998influence}. Researchers found that maternal hair levels were significantly associated with poorer scores on IQ, language development, and gross motor skills on children at age 6 years \citep{kjellstrom1989physical}.

\underline{Faroe Islands Study} A birth cohort study carried out in the Faroe Islands examined the associations between prenatal MeHg exposure (primarily due to the maternal consumption of whale meat and blubber) and infant/child neurodevelopment \citep{grandjean1992impact,grandjean1997cognitive}. During 1986-1987 (21 months), a total of 1,022 newborns were recruited; cord blood samples, maternal scalp hair samples, and nutrition habits (including the consumption frequency of fish or pilot whale) were obtained \citep{grandjean1992impact}. The median cord blood Hg was 24.2 \({\mu}\)g/L and the median maternal hair Hg during pregnancy was 4.48 \({\mu}\)g/g \citep{grandjean1992impact}. In 1993, when children were 7 years old, \cite{grandjean1992impact} conducted a comprehensive neurobehavioral examination on 917 children of this birth cohort. The authors then found that cord blood Hg concentrations were negatively and significantly associated with test scores of three major cognitive functions: language, attention, and memory \citep{grandjean1992impact}. During 2000-2001, 878 children (at 14 years of age) of the Faroe Islands cohort were evaluated for neurobehavioral development, and most of the neuropsychological tests used in this study were similar to the tests applied in the previous study on children at age 7 \citep{debes2006impact}. The results indicated that higher cord blood Hg levels were significantly associated with lower scores in neurodevelopmental tests \citep{debes2006impact}. Moreover, the authors compared the findings at age 7 and at age 14 and reported that Hg-related dysfunctions were similar between these two studies \citep{debes2006impact}.

\underline{Seychelles Islands Study} The Seychelles Child Development study was aimed to investigate the developmental influences of prenatal MeHg exposure in the Seychellois, who consumed a wide variety of marine fish/shellfish on a daily basis \citep{davidson2006prenatal}. During 1989 and 1990, a total of 779 infant-mother pairs were recruited in this study; the mean Hg levels in maternal hair during pregnancy was 6.1 \({\mu}\)g/g \citep{marsh1994seychelles}. Since enrollment, researchers have evaluated the relationships between maternal hair Hg levels and neurodevelopmental endpoints when the children were 6.5 months to 10.7 years old \citep{davidson1998effects,davidson2006prenatal,davidson2008association,myers2003prenatal}. However, findings have so far offered no convincing evidence of impairment on fetal neurodevelopment due to maternal MeHg exposure during pregnancy through the consumption of marine fish \citep{davidson2006prenatal,davidson2008association}. 

In these three epidemiologic studies (New Zealand, Faroe Islands, and Seychelles Islands), the mean Hg levels in maternal hair during pregnancy were similar (4-6 \({\mu}\)g/g). However, the adverse neurodevelopmental effects on infants/children were observed in studies from New Zealand and the Faroe Islands, but not in the mother-child cohort from Seychelles. These inconsistencies may be due to the benefit of fish/shellfish consumption. Although fish/shellfish consumption is the main MeHg exposure pathway, fish and shellfish are significant sources of beneficial nutrients for fetal development, such as long chain polyunsaturated fatty acids (LCPUFAs), iodine, selenium, and Vitamin D \citep{national2000toxicological}. N-3 polyunsaturated fatty acids (n-3 PUFAs) includes docosahexaenoic acid (DHA), eicosapentaenoic acid (EPA), and arachidonic acid (AA) \citep{siriwardhana2012health}, which are beneficial to fetal neurodevelopment \citep{birch2007visual}. For example, in a U.S. infant cohort (n=135), the results of linear regression models showed that maternal fish consumption frequency during pregnancy was positively related with the score of visual recognition memory testing at 6 months of age after adjusting for participant characteristics and maternal hair Hg levels \citep{oken2005maternal}. Insufficient intake of n-3 PUFAs during pregnancy was found to be associated with impaired fetal development, such as intrauterine growth retardation and delayed neurodevelopment \citep{hibbeln2007maternal}.

Similar findings were also reported by other studies. For example, in a cohort of 7,421 British children, no adverse relationships were observed between cord blood THg and child's cognitive development at age 15 months (the MacArthur Communicative Development Inventory test) and at age 18 months (the Denver Developmental Screening Test) in adjusted models (including fish consumption); and maternal fish consumption during pregnancy was found to be positively associated with test scores \citep{daniels2004fish}. Likewise, a Spanish study (n=1683) found that cord blood THg was not associated with the mental and psychomotor development which were evaluated at age 14 months using the Bayley Scales of Infant Development \citep{llop2012prenatal} in adjusted models (including fish consumption). However, a study conducted in the U.S. (n=604) found positive associations between maternal hair Hg levels and attention-deficit/hyperactivity disorder at age 8 years in both unadjusted and adjusted models (including fish consumption) \citep{sagiv2012prenatal}.

Thus, the fetal developmental influences of chronic, low-level prenatal exposure to MeHg through maternal consumption of fish/shellfish during pregnancy is quite complex \citep{rice2004us,cohen2005quantitative}. Potential adverse effects of prenatal MeHg exposure on fetal development due to maternal consumption of fish/shellfish might be alleviated by the contribution of micronutrients in fish/shellfish tissue. The balance between nutritional benefits and risks of maternal consumption of fish/shellfish during pregnancy has been a challenge for government agencies and experts \citep{who1990mehg,national2000toxicological,usep2014fish}. 

\subsection{Rice and MeHg exposure}

Recent evidence has indicated that rice is potentially a significant source of dietary MeHg exposure \citep{li2010methylmercury,rothenberg2014rice}. A review about the rice MeHg and exposure \citep{rothenberg2014rice} stated that rice cultivation, from planting to harvest, helps make flooded rice paddies to be \(\text{Hg}^{2+}\)-methylation sites. In rice paddies, rice is cultivated under standing water (from irrigation or rain), serving anoxic conditions for \(\text{Hg}^{2+}\)-methylation \cite{rothenberg2014rice}. MeHg can be produced in sediments in an anaerobic microbial process driven by the sulfate-reducing bacteria, which convert less toxic inorganic (\(\text{Hg}^{2+}\)) into its more toxic organic form, MeHg \citep{rothenberg2014rice}. Then, MeHg is likely translocated from the rice paddies sediment to rice grain \citep{rothenberg2014rice}. 

However, compared to fish/shellfish, rice does not contain the same beneficial micronutrients that fish/shellfish has, such as n-3 PUFAs \citep{rothenberg2011low}. Prenatal MeHg exposure through maternal rice consumption may potentially increase adverse health effects on the developing fetus \citep{rothenberg2011low}.

\section{Knowledge gaps}

Given that rice is a staple food for nearly half of the world's seven billion people \citep{mohanty2013trends}, rice consumption may be an important dietary source of MeHg, with widespread impact. However, most of the current pregnant cohort studies on dietary MeHg exposure focused on the fish/shellfish consumption. To the best of our knowledge, all current pregnant cohort studies about MeHg exposure through rice ingestion were conducted in Hg mining areas where Hg concentrations in rice were relatively high \citep{morishita1982mercury,rothenberg2013prenatal}. 

\section{Research aims}

The main objectives of this research were to determine the magnitude of in utero exposure to MeHg and to investigate the possible health influences of prenatal MeHg exposure on neonatal anthropometrics in a cohort of pregnant women who consumed rice as a staple food. This study was conducted through the implementation of a birth cohort study at the Maternal and Children Hospital in Daxin County, Guangxi Province, China. Maternal Hg biomarkers and information for the mothers and newborns were collected from 2013 through 2014.

In Chapter 2, we evaluate the prenatal MeHg exposure through rice and fish/shellfish ingestion and investigate the possible demographic indicators of dietary MeHg intake. Chapter 3 presents the results of the associations between Hg concentrations in maternal hair and blood and neonatal outcomes, including birth weight, birth length, head circumference, and ponderal index.

The results of this research provide information on the magnitude of MeHg exposure among pregnant women in a rural inland area of China, where rice is the staple food and a significant source of MeHg exposure. Meanwhile, the findings of relationships between prenatal MeHg and neonatal outcomes are useful in understanding the potential influences on developing fetuses due to MeHg exposure through rice ingestion in rice-eating populations. 


