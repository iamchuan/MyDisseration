\textbf{Abstract}

Introduction: A few studies have indicated that human methylmercury exposure occurred through rice ingestion. However, little is known about the magnitude of dietary methylmercury exposure through rice ingestion in pregnant women.

Objectives: The main objective was to evaluate dietary methylmercury exposure through rice and fish/shellfish consumption among pregnant women living in rural southern China, where rice is a staple food. 

Methods: A total of 398 pregnant mothers were recruited at parturition in an inland rural area of Guangxi, China, where mercury contamination was considered minimal. Mothers donated scalp hair and blood samples, filled out a questionnaire on demographic characteristics and food consumption during the third trimester. Total mercury and/or methylmercury concentrations were measured in maternal hair and blood, in rice samples from each participant's home, and in freshwater fish tissue purchased from local markets. Additional fish/shellfish mercury concentrations (including marine fish, eel, shrimp, crab, snails, and other shellfish) were obtained from a comprehensive literature search. The dietary methylmercury intake through rice and fish/shellfish ingestion was calculated using mothers's responses to the food frequency questionnaire and the mercury concentrations of rice and fish/shellfish. 

Results: 82\% of the participants (n=328) reported consuming rice daily, while 43\% (n=172) reported rarely or never consuming fish/shellfish. The estimated daily total of methylmercury intake averaged 1.2 \({\mu}\)g/day, including 71\% through rice consumption and 29\% through fish/shellfish consumption. The geometric means of hair total mercury, hair methylmercury, and blood total mercury were 0.41 \({\mu}\)g/g, 0.27 \({\mu}\)g/g, and 1.3 \({\mu}\)g/g, respectively, which were comparable to values for several pregnant cohorts from other countries with low-level mercury exposure mainly through fish/shellfish consumption. Positive correlations were observed between dietary MeHg intake through rice or fish/shellfish consumption and all three mercury biomarkers; however, trends were significant and relatively strong for rice MeHg intake (Spearman's rho: 0.18-0.21, p<0.01) and non-significant for fish/shellfish MeHg intake (Spearman's rho: 0.04-0.08, p>0.1). 

Conclusions: Among pregnant mothers living in Daxin, rice ingestion contributed to prenatal methylmercury exposure, more so than fish/shellfish ingestion.

\newpage